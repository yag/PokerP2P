%!TEX root = /Users/regis/code/M1/be/projet_be.tex

\minitoc
\clearpage

\section*{Diagramme de cas d'utilisation}\vspace{3em}
\begin{figure}[h]
	\centering \includegraphics[width=.5\textwidth]{uml/usecase/usecase.pdf}
\end{figure}
\clearpage

%%-------------------------  USE CASE : Créer partie  ------------------------%%

\begin{usecase}{Créer partie}
	
	\begin{presentation}
		\actor Utilisateur
		\goal ouverture d'une table de jeu selon les paramètres définis 
		      (cf règles de gestion).
		
		\begin{precondition}
			\aucune
		\end{precondition}
		
		\begin{postcondition}
			\aucune
		\end{postcondition}
		
		\begin{includeuc}
			\aucun
		\end{includeuc}
	\end{presentation}
	
	\begin{scenario}
		\start demande la création d'une table de jeu
			\system demande les paramètres (cf règles de gestion)
		\user fournit les paramètres
			\system vérifie la validité de ces données et crée la partie
	\end{scenario}
	
	\begin{alternative}
		\nomalt Échec vérification
			\condition à l'étape 4, si la validation d'un des paramètres échoue
			
			\begin{alt}
				\item GOTO étape 3
			\end{alt}
	\end{alternative}
	
	\begin{regles}
		\nom Les paramètres doivent être valides  :
		
		\begin{enumerate}
			\regle la cave initiales, à partir de 10 \$, sans limite
			\regle temps de décision entre 1 et 15 secondes
			\regle nombre de joueurs requis pour commencer à jouer, entre 2 et 8
			\regle nombre de joueurs maximum à la table, entre 2 à 8
			\regle SB et BB de départ
			\regle temps entre les blind-up (au moins une minute, sans limite)
			\regle schéma d'augmentation des SB, BB et Ante
		\end{enumerate}

	\end{regles}
	
	\sequencediagram{uml/usecase/TableCreationSequenceDiagram.pdf}
	                {Créer partie}
	
\end{usecase}

%%----------------------------------------------------------------------------%%

%%----------------------  USE CASE : Rejoindre Partie  -----------------------%%

\begin{usecase}{Rejoindre partie}
	
	\begin{presentation}
		\actor Utilisateur
		\goal accèder à une table de jeu de manière à pouvoir jouer, mais sans 
		      interrompre une main en cours, donc en devenant spectateur.
		
		\begin{precondition}
			\aucune
		\end{precondition}
		
		\begin{postcondition}
			\condition spectateur
			\condition en attente de fin de main pour pouvoir postuler pour 
			           devenir joueur si une place est libre
		\end{postcondition}
		
		\begin{includeuc}
				\aucun
		\end{includeuc}
		
	\end{presentation}

	\begin{scenario}
		\start demande l'accès à une table de jeu
		\system vérifie l'existence de la table de jeu et rejoint cette table
	\end{scenario}

	\begin{alternative}
		\nomalt Erreur table
		\condition à l'étape 2, si la table n'existe pas

		\begin{alt}
			\system prévient l'acteur et celui-ci reste spectateur
		\end{alt}

		\nomalt Erreur banni
		\condition à l'étape 2, si le serveur refuse la connexion car le 
		joueur a été banni

		\begin{alt}
			\user est averti et GOTO 1
		\end{alt}

	\end{alternative}
	
	\sequencediagram{uml/usecase/TableJoinSequenceDiagram.pdf}
	                {Rejoindre Partie}
	
\end{usecase}

%%----------------------------------------------------------------------------%%

%%-------------------  USE CASE : Demander Devenir Joueur  -------------------%%

\begin{usecase}{Demander Devenir Joueur}
	
	\begin{presentation}
		\actor Spectateur
		\goal passer du statut de spectateur à celui de joueur au sein de la 
		      table de jeu actuelle, et pouvoir participer à la partie en cours.
		
		\begin{precondition}
			\condition être spectateur d'une table de jeu ouverte
			\condition il n'y a pas de main en cours
		\end{precondition}
		
		\begin{postcondition}
			\condition joueur
		\end{postcondition}
		
		\begin{includeuc}
			\aucun
		\end{includeuc}
	\end{presentation}
	
	\begin{scenario}
		\start demande à devenir joueur
			\system demande au serveur l'autorisation et l'utilisateur devient 
			        joueur
	\end{scenario}
	
	\begin{alternative}
		\nomalt Impossible Joueur
			\condition à l'etape 2, si le serveur refuse que le spectateur 
			           devienne joueur
		
		\begin{alt}
			\system prévient l'acteur et celui-ci reste spectateur
		\end{alt}
	\end{alternative}
	
	\begin{regles}
		\nom Le spectateur peut ne pas pouvoir devenir joueur pour plusieurs 
		     raisons :
		
		\begin{enumerate}
			\regle il n'y a pas de place
			\regle d'autres spectateurs ont demandé à devenir joueurs avant lui
		\end{enumerate}
		
	\end{regles}
	
	\sequencediagram{uml/usecase/BecomingPlayerSequenceDiagram.pdf}
	                {Demande devenir joueur}
	
\end{usecase}

%%----------------------------------------------------------------------------%%

%%-----------------------  USE CASE : QUITTER PARTIE  ------------------------%%

\begin{usecase}{Quitter Partie}
	
	\begin{presentation}
		\actor Spectateur ou Joueur
		\goal se déconnecter volontairement d'une table de jeu
		
		\begin{precondition}
			\condition avoir rejoint cette table
			\condition si l'acteur initiateur est un joueur, qu'il n'y ait pas 
			           de main en cours
		\end{precondition}
		
		\begin{postcondition}
			\condition état initial du systeme
		\end{postcondition}
		
		\begin{includeuc}
			\aucun
		\end{includeuc}
	\end{presentation}
	
	\begin{scenario}
		\start demande à être déconnecté de la table
			\system revient à son etat initial
	\end{scenario}
	
	\begin{alternative}
		\nomalt Erreur Main
			\condition à l'étape 2, si une main est en cours et que l'acteur 
			           initiateur y joue
			
		\begin{alt}
			\system indique au joueur qu'il lui faut attendre la fin de la main
		\end{alt}
	\end{alternative}
	
	\sequencediagram{uml/usecase/DeconnexionSequenceDiagram.pdf}
	                {Quitter partie}
	
\end{usecase}

%%----------------------------------------------------------------------------%%

%%-----------------------  USE CASE : Envoyer Message  -----------------------%%

\begin{usecase}{ Envoyer Message}
	
	\begin{presentation}
		\actor Spectateur ou Joueur
		\goal communiquer avec d'autres joueurs/spectateurs de la table par des 
		      messages textuels.
		
		\begin{precondition}
			\condition avoir rejoint une table
		\end{precondition}
		
		\begin{postcondition}
			\aucune
		\end{postcondition}
		
		\begin{includeuc}
			\aucun
		\end{includeuc}
	\end{presentation}
	
	\begin{scenario}
		\start demande le post d'un message dans le chat
			\system envoie le message au serveur et l'affiche dans le chat
	\end{scenario}
	
	\begin{alternative}
		\nomalt Envoi impossible
			\condition À l'étape 2, si l'utilisateur n'a pas le droit de poster 
			           le message
			
			\begin{alt}
				\system l'en informe et n'envoie pas le message
			\end{alt}
	\end{alternative}
	
	\begin{regles}
		\nom L'utilisateur peut ne pas avoir le droit de poster de message si :
		
		\begin{enumerate}
			\regle il a déjà posté 3 messages d'affilée
			\regle il a déjà posté 3 messages en moins d'une minute
			\regle le message contient certains mots interdits (injures, par exemple)
		\end{enumerate}
	\end{regles}
	
	\sequencediagram{uml/usecase/ChatSequenceDiagram.pdf}
	                {Envoyer message}
		
\end{usecase}

%%----------------------------------------------------------------------------%%

%%---------------------  USE CASE : Devenir Spectateur  ----------------------%%

\begin{usecase}{Devenir Spectateur}
	
	\begin{presentation}
		\actor Joueur
		\goal arrêter de jouer à la table mais observer le déroulement 
		      de la partie en cours
		
		\begin{precondition}
			\condition avoir rejoint une table
			\condition être joueur à cette table
			\condition qu'il n'y a pas une main en cours
			\condition qu'il y ait au moins 2 autres joueurs à la table
		\end{precondition}
		
		\begin{postcondition}
			\condition spectateur
		\end{postcondition}
		
		\begin{includeuc}
			\aucun
		\end{includeuc}
	\end{presentation}
	
	\begin{scenario}
		\start demande à être spectateur
			\system annonce devenir spectateur et le devient
	\end{scenario}
	
	\sequencediagram{uml/usecase/BecomingSpectatorSequenceDiagram.pdf}
	                {Devenir spectateur}
	
\end{usecase}

%%----------------------------------------------------------------------------%%

%%-------------------------  USE CASE : Se Coucher  --------------------------%%

\begin{usecase}{Se Coucher}
	
	\begin{presentation}
		\actor Joueur
		\goal arrêter de jouer pour la main en cours
		
		\begin{precondition}
			\condition être en train de jouer une main
		\end{precondition}
		
		\begin{postcondition}
			\condition en attente de la fin de la main courante
		\end{postcondition}

		\begin{includeuc}
			\aucun
		\end{includeuc}
	\end{presentation}
	
	\begin{scenario}
		\start demande à se coucher
			\system annonce se coucher et attend la fin de la main en cours
	\end{scenario}
	
	\sequencediagram{uml/usecase/FoldSequenceDiagram.pdf}
	                {Se coucher}
	
\end{usecase}

%%----------------------------------------------------------------------------%%

%%---------------------------  USE CASE : Suivre  ----------------------------%%

\begin{usecase}{Suivre}

	\begin{presentation}
		\actor Joueur
		\goal égaliser la mise maximale actuelle au cours d'une main
		
		\begin{precondition}
			\condition être en train de jouer une main
		\end{precondition}
		
		\begin{postcondition}
			\condition en attente du nouveau tour de mise (suivant ou courant à égaliser)
		\end{postcondition}
		
		\begin{includeuc}
			\aucun
		\end{includeuc}
	\end{presentation}
	
	\begin{scenario}
		\start annonce qu'il suit la mise actuelle
			\system vérifie que la somme est valide, fait état de l'action et attend le prochaine tour de parole
	\end{scenario}
	
	\begin{alternative}
		\nomalt ERREUR mise
			\condition à l'étape 2, si la somme n'est pas valide
			
		\begin{alt}
			\system annonce qu'il est tapis
		\end{alt}
	\end{alternative}
	
	\begin{regles}
		\nom Le joueur possède l'argent qu'il mise
	\end{regles}
	
	\sequencediagram{uml/usecase/CallingSequenceDiagram.pdf}
	                {Suivre}
	
\end{usecase}

%%----------------------------------------------------------------------------%%

%%--------------------------  USE CASE : Relancer  ---------------------------%%

\begin{usecase}{Relancer}
	
	\begin{presentation}
		\actor Joueur
		\goal augmenter la mise maximale actuelle au cours d'une main
		
		\begin{precondition}
			\condition être en train de jouer une main
		\end{precondition}
		
		\begin{postcondition}
			\condition en attente du nouveau tour de mise (suivant ou courant à égaliser)
		\end{postcondition}
		
		\begin{includeuc}
			\aucun
		\end{includeuc}
	\end{presentation}
	
	\begin{scenario}
		\start demande à relancer d'une certaine somme
			\system vérifie que la somme est valide, fait état de la relance et attend le prochain tour de parole
	\end{scenario}
	
	\begin{alternative}
		\nomalt ERREUR mise
			\condition à l'étape 2, si la somme n'est pas valide
			
		\begin{alt}
			\system l'annonce au joueur et GOTO 1
		\end{alt}
	\end{alternative}
	
	\begin{regles}
		\nom Validation de la mise :
		\begin{enumerate}
			\regle le joueur possède l'argent qu'il mise
			\regle la somme est un multiple de la blind
			\regle la somme est un multiple de la précédente mise
		\end{enumerate}
	\end{regles}
	
	\sequencediagram{uml/usecase/RaiseSequenceDiagram.pdf}
	                {Relancer}
	
\end{usecase}

%%----------------------------------------------------------------------------%%

%%-------------------------  USE CASE : Faire Tapis  -------------------------%%

\begin{usecase}{Faire Tapis}
	
	\begin{presentation}
		\actor Joueur
		\goal égaliser ou augmenter la mise maximale actuelle en jouant tout son argent
		
		\begin{precondition}
			\condition être en train de jouer une main
		\end{precondition}
		
		\begin{postcondition}
			\condition suivant le résultat de la main, devenir spectateur ou rester joueur
		\end{postcondition}
		
		\begin{includeuc}
			\aucun
		\end{includeuc}
	\end{presentation}
	
	\begin{scenario}
		\start demande à faire tapis 
			\system fait état du tapis et attend le prochain tour de parole
	\end{scenario}
	
	\begin{alternative}
		\nomalt ERREUR mise
			\condition à l'étape 2, si la somme n'est pas valide
			
		\begin{alt}
			\system l'annonce au joueur et GOTO 1
		\end{alt}
	\end{alternative}
	
	\begin{regles}
		\nom Séparation des pots
	\end{regles}
	
	
\end{usecase}

%%----------------------------------------------------------------------------%%

%%-------------------  USE CASE : Consulter Statistiques  --------------------%%

\begin{usecase}{Consulter Statistiques}
	
	\begin{presentation}
		\actor Joueur
		\goal obtenir ses propres statistiques ainsi que celles des joueurs jouant à la table
		
		\begin{precondition}
			\aucune
		\end{precondition}
		
		\begin{postcondition}
			\aucune
		\end{postcondition}
		
		\begin{includeuc}
			\aucune
		\end{includeuc}
	\end{presentation}
	
	\begin{scenario}
		\start demande l'obtention des statistiques d'un joueur (peut-être lui-même)
			\system lui fournit ces informations
	\end{scenario}
	
	\sequencediagram{uml/usecase/StatsSequenceDiagram.pdf}
	                {Consulter statistiques}
	
\end{usecase}