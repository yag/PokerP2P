%!TEX root = /Users/regis/code/M1/be/projet_be.tex

\section{Introduction}

\subsection{\`A propos de ce document}
Dans le cadre de la réalisation d'une interface de poker multijoueur en 
pair-à-pair décentralisé, ce document constitue la première étape de 
modélisation du projet. 
Il contient :
{\renewcommand\labelitemi{\textbullet}
\begin{itemize}
	\item Les diagrammes d'activités pour la modélisation des processus métier.
	\item Un diagramme de classe pour la modélisation du domaine métier.
	\item La spécification du protocole de communication entre les clients et le
	      serveur.
	\item La spécification complète des besoins fonctionnels. 
\end{itemize}}

\section{Contraintes techniques}

{\renewcommand\labelitemi{\textbullet}
\begin{itemize}
	\item Le langage de programmation utilisé est JAVA (version $>= 1.5$) pour 
	      la partie client, comme la partie serveur.
	\item La couche de communication sera implémentée en utilisant le protocole
		  JAVA RMI (Remote Method Invocation), ce qui suppose que les parties
		  clientes et serveur sont compatibles au niveau de leurs interfaces.
	\item Le programme livré devra être portable et fonctionner dans les
	      environnements de type Windows ou Unix.
\end{itemize}}

\section{Cahier de spécifications}

\subsection{Notre démarche}

Le sujet du bureau d'étude étant relativement concis, nous avons décidés dans un
premier temps de produire un cahier de spécifications complet qui nous servira
de base pour les travaux à venir.
Dans ce cahier sont regroupés l'ensemble des règles métier concernant le poker,
principalement les règles du jeu, puis quelques considérations techniques sur 
des fonctionnalités à ajouter à notre logiciel.

