%!TEX root = /Users/regis/code/M1/be/projet_be.tex

\section{Modélisation des processus métier}
Afin de facilité la compréhension du diagramme d'activité, nous avons décidé de 
le scinder en trois diagrammes compléementaires. Pour des raisons évidentes de
mises en page, les diagrammes UML se trouve en fin de ce chapitre sur des pages
dédiées.

\subsection{Déroulement général d'une partie}
Ce premier diagramme présente le déroulement général d'une partie. Tant qu'il 
reste au moins deux joueurs à la table, on continue à faire des parties.

\subsection{Désignation des rôles, mises obligatoires, distribution des cartes}
Ce second diagramme explicite la phase de désignation des rôles (dealer, SB \& 
BB), de pose des mises obligatoires (SB, BB \& \textsc{Ante}) ainsi que la 
distribution des cartes au début de chaque partie. On notera que la distribution
des cartes se fait dans le sens horaire, une carte par joueur, jusqu'à que 
chaque joueur à la table est reçu ses deux cartes privatives.

\subsection{Parler}
Ce dernier diagramme détaille l'activité \textit{parler} du premier diagramme.
Il est à noter que l'activité \textit{miser} regroupe :

\begin{itemize}
	\item suivre ou \textit{call} qui consiste à égaliser la mise précédente.
	\item relancer ou \textit{rise} qui consiste à poser au moins le double de 
	      la mise précédente.
	\item miser ou \textit{bet} qui consiste à poser au moins la valeur de la 
	      Big Blind.
	\item passer ou \textit{check} qui est possible si aucune mise n'à été 
	      posée et qui permet au joueur de \og temporiser \fg.
	\item se coucher ou \textit{fold} qui consiste à jeter ses cartes et quitter 
	      la partie en cours. Cela à pour conséquence la perte de l'intégralité 
	      des jetons engagés par le joueur jusque là.
\end{itemize}

\subsection{Remarques générales}
\begin{itemize}
	\item Ces diagrammes proposent la configuration pré-flop. Si on considère 
	      les autres tours de mise, ce sera la petite blind qui commencera à 
	      parler.
	\item Pour une configuration à deux joueurs, on inversera les rôles entre 
	      Small Blind et Big Blind.
	\item L'utilisation du joueur Lambda n'est pas anodine. Il permet de 
	      représenter un joueur à la table qui n'a aucun rôle particulier et 
	      donc par extention, il représente l'ensemble des autres joueurs à la 
	      table.
\end{itemize}

\section{Modélisation du domaine métier}
\subsection{Diagramme}
DIAGRAMME
\subsection{Associations}
\begin{itemize}
	\item En représentant la classe \textit{Mise} comme l'association entre un 
	      \textit{Joueur} et un \textit{Tour}, on met en évidence qu'une mise 
	       n'est possible que si un joueur \textit{mise} pendant un tour.
	\item Une \textit{Main} n'existe que si un \textit{Croupier} 
	      \textit{distribue} des cartes à un joueur. C'est donc pour cela que 
	      nous représentont un la classe \textit{Main} comme une classe 
	      associative.
	\item Un \textit{Pot} n'est que l'accumulation de \textit{Mises} durant une 
	      \textit{Partie} à chaque \textit{Tour}. Aucun pot ne peut donc pas 
	      exister sant l'intéraction d'une partie et d'un ensemble de tours. 
	      C'est donc pour cela que l'on représente un \textit{Pot} comme une 
	      classe associative entre \textit{Tour} \& \textit{Partie}.
\end{itemize}
\subsection{Relations de Généralisation}
\begin{itemize}
	\item \textsc{Tour} est une généralisation du \textsc{Pré-Flop}, du 
	      \textsc{Flop}, du \textsc{Turn} \& de la \textsc{River}.
	\item \textsc{Joueur} est généralisation du \textsc{Dealer}, de la 
	      \textsc{Small Blind}, de la \textsc{Big Blind}, et du joueur 
	      \textsc{Lambda}. Notez que le joueur \textsc{Lambda} désigne un joueur
	      quelconque mais différent de la \textsc{Small Blind}, de la
	      \textsc{Big Blind}, ou encore du \textsc{Dealer}
\end{itemize}

\begin{figure}[ht]
	\includegraphics[width=\textwidth]{uml/metier/Activite1.pdf}
\end{figure}


\begin{figure}[ht]
	\centering \includegraphics[width=1.50\linewidth, angle=90]{uml/metier/Activite2.pdf}
\end{figure}

\begin{figure}[ht]
	\centering \includegraphics[width=1.50\linewidth, angle=90]{uml/metier/Activite3.pdf}
\end{figure}

\begin{figure}[ht]
	\centering \includegraphics[width=1.50\linewidth, angle=90]{uml/metier/Domaine.pdf}
\end{figure}
