%!TEX root = /Users/regis/code/M1/be/projet_be.tex

Ici, nous allons détailler quelques unes des nombreuses expressions utilisées dans le monde du poker. Nous nous attarderons que sur celles qui sont les plus utilisées et sur lesquelles on pourra être amené à rencontrer tout au long du développement du projet.\\

\begin{df}{add on}
		Le add on désigne le montant supplémentaire à payer pour acheter un nouveau tapis et ainsi revenir à la table.
\end{df}

\begin{df}{bet}
	Bet est l'expression anglaise signifiant ``miser''.
\end{df}

\begin{df}{call}
	Call est l'expression anglaise signifiant ``suivre''.
\end{df}

\begin{df}{cartes publiques}
	Les cartes publiques sont les cartes qui constituent le tableau. Elle sont dites publiques car tous les joueurs peuvent les voir. Elles appartiennent à l'ensemble des joueurs en jeu à la table.	
\end{df}


\begin{df}{cartes privatives}
	Les cartes privatives sont les deux cartes qui constituent la main du joueur. Il est le seul à les connaitre, c'est pour cela qu'on les appelles cartes privées ou privatives.
\end{df}

\begin{df}{rise}
	Rise est l'expression anglaise signifiant ``relancer''.
\end{df}

\begin{df}{Call}
	Call est l'expression anglaise signifiant ``suivre''.
\end{df}

\begin{df}{Check}
	Check est l'expression anglaise signifiant ``passer''.
\end{df}

\begin{df}{Fold}
	Fold est l'expression anglaise signifiant ``se soucher''.
\end{df}

\begin{df}{Relancer}
	Lorsqu'un joueur de la table a misé, un autre joueur peut relancer ou \textit{``rise''}. Cela signifie qu'il décide de mettre plus de jetons que le précédent. Sa relance doit être au minimum deux fois plus importante que la mise qu'il relance.
\end{df}

\begin{df}{Suivre}
	Suivre ou \textit{``Call''} est l'action de payer la mise qui est sur la table. Si un joueur mise \$20, les autres devrons payer \$20 pour suivre et donc rester dans la partie.
\end{df}

\begin{df}{Miser}
	Miser ou \textit{``Bet''} est l'action de poser des jetons sur la table et contraindre les autres joueurs à payer pour rester en jeu. L'ensemble des mises constituent le pot de la table qui sera remporté (ou partagé) par le(s) vainqueur(s) de la main.
\end{df}

\begin{df}{Se Coucher}
	Se Coucher ou \textit{``Fold''} ou encore \textit{``Folder''} est l'action de jeter ses cartes et quitter la partie en cours. Les sommes engagées jusqu'à cette décision sont définitivement perdues.
\end{df}

\begin{df}{Under the Gun}
	L'expression \textit{``Under the Gun''} désigne la position du joueur qui parle juste après la BB. Cette position est appelée ainsi parce que ce joueur sera le premier à parler en pré-flop et prendra donc un maximum de risques.
\end{df}

\begin{df}{Turn}
	Le \textit{Turn} est la carte distribuée après le tour de mises (ou tour d'enchères) du Flop.
\end{df}

\begin{df}{River}
	La \textit{River} est la dernière cartes des cinq cartes du tableau a être distribuée. Après le tour d'enchère qu'il s'en suit, les joueurs restant devront abattre leurs cartes afin de trouver le ou les vainqueurs de la partie.
\end{df}

\begin{df}{Abattage}
	L'abattage est l'action de montrer ses cartes. On abats ses cartes que s'il reste des joueurs à l'issue du tour d'enchères de la River ou lorsqu'on désire montrer ses cartes à la fin d'une partie que l'on a gagné. Le premier à montrer ses cartes est le premier à avoir misé, les autres joueurs déciderons de montrer ou non leur main et cela dans le sens horaire.
\end{df}

\begin{df}{Tour de mises}
	Le tour de mises ou tour d'enchêres our encore tour de paroles est le moment de paroles des joueurs juste après qu'ils est reçus une ou plusieurs nouvelles cartes et qui ne se termine que lorsque tous les joueurs se sont couchés sauf un ou lorsque toutes les mises on été égalisées.
\end{df}

\begin{df}{Tour d'enchères}
	Synonyme de ``Tour de mises''.
\end{df}

\begin{df}{Tour de paroles}
	Synonyme de ``Tour de mises''.
\end{df}

\begin{df}{Flop}
	Après le premier tour de paroles, les joueurs restant reçoivent trois cartes publiques supplémentaires pour compléter leur jeu. Ces trois cartes sont appelées ``Flop''. Après la distribution du Flop, un autre tour d'enchère se met en place.
\end{df}

\begin{df}{Pré-Flop}
	Le tour d'enchères avant la distributrion du Flop est appelée ``Pré-Flop''.
\end{df}

\begin{df}{Kicker}
	Toutes mains au Texas Hold'em est constitué des cinqs meilleures cartes parmis les deux privatives et les 5 publiques qui constituent le tableau. En complément d'une combinaison, on désigne le kicker : la carte la plus forte afin de départager deux mains équivalente. Par exemple, une paire de Roi kicker As remporte sur une paire de Roi kicker Dame. Une couleur kicker roi (ou hauteur roi) est inférieure à une couleur kicker as.
\end{df}

\begin{df}{Big Blind}
	La Big Blind ou BB désigne à la fois une position à la table et une mise obligatoire. La Big Blind est le joueur qui se situe deux positions après celle du dealer. Ce joueur là doit payer obligatoirement un mise prédéfinie s'il veut recevoir des cartes et participer à la partie suivante. S'il ne paye pas la Big Blind est dite ``avalée'' et le joueur est mis sit-out.
\end{df}

\begin{df}{Sit-Out}
	Lorsqu'un joueur ne désire pas quitter sa place à la table mais ne veut plus joueur pendant un certain temps, il peut désirer de se mettre sit-out. Dans ce cas là, il conserve sa place mais il paye toutes les mises obligatoires de la table (SB, BB \& \textsc{Ante}) et il est distribué pour être automatiquement foldé. Un joueur est automatiquement mis dans cette position s'il est déconnecté ou si ce dernier n'as plus de jetons.
\end{df}

\begin{df}{Sit-In}
	Lorsqu'un joueur s'est mis Sit-out, le fait qu'il reviennent en jeu à la table est appelé Sit-In. Pour revenir en jeu, le joueur doit attendre la fin de la partie en cours  et payer un mise de bienséance équivalente à la Big Blind (ou rien s'il est en position de Big Blind).
\end{df}

\begin{df}{Ante}
	\textsc{L'Ante} est une mise obligatoire aparraissant au bout d'un certain temps de jeu que tout joueur doit payer en début de main pour recevoir sa main. Cette mise est en général faible par rapport à la SB et la BB.
\end{df}

\begin{df}{Tapis}
	Le tapis d'un joueur ou \textit{``stack''} en anglais  est l'ensemble des jetons que possède ce dernier. En cours de jeu, si un joueur s'écrit \textit{``Tapis''} ou \textit{``All-In''} cela signifie que se dernier désire engagé tous ces jetons dans la partie. 
\end{df}

\begin{df}{All-in}
	cf. Tapis
\end{df}

\begin{df}{Stack}
	Stack est l'expression anglaise signifiant ``Tapis''.
\end{df}

\begin{df}{Dealer}
	Le dealer est la position du joueur à la table qui distribue les cartes. Dans la pratique aucun joueur ne touche aux cartes, la distribution étant effectuée par un croupier. Le dealer devient donc un rôle totalement symbolique. C'est lui qui à le bouton dealer. 
\end{df}

\begin{df}{Board}
	Board est l'expression anglaise signifiant ``Tableau''.
\end{df}

\begin{df}{Tableau}
	Le tableau est l'ensemble des zéros à cinq cartes publiques suivant que le flop, la turn ou la river aient été distribués. 
\end{df}

\begin{df}{Pot}
	Le pot est le montant total des mises engagées par les joueurs de la tables depuis le début de la main. Il est modélisé par un ensemble de jetons.
\end{df}

\begin{df}{Sur relancer}
	Sur-relancer est le fait de relancer une relance déjà effectuée sur la mise de départ.
\end{df}

\begin{df}{nuts}
	Lorsqu'un joueur à ce que l'on appelle \textit{``touché les nuts''} (pas de mauvais esprits, s'il vous plait), c'est qu'il a possède la plus grosse main possible pour un tableau de donné. Il est ainsi sûr de gagner la partie quoi qu'il arrive.
\end{df}

\begin{df}{Bouton}
	La position du bouton est la position du dealer en référence au fait que ce dernier possede le bouton dealer.
\end{df}

\begin{df}{Buy-In}
	Le Buy-In est le prix du tapis initial que doit payer un joueur afin de prendre position à la table.
\end{df}

\begin{df}{Donneur}
	Donneur est l'expression française désignant le Dealer.
\end{df}

\begin{df}{Rebuy}
	Le rebuy est l'expression anglaise pour la ``Recave''
\end{df}

\begin{df}{Recave}
	Lorsqu'un joueur n'a plus de jetons, il peut acheter un nouveau tapis avec un éventuel add-on sous certaines conditions.
\end{df}