	\documentclass[a4paper]{article}

\usepackage[francais]{babel}
\usepackage[T1]{fontenc}
\usepackage[utf8]{inputenc}
\usepackage[top=2cm, bottom=2cm, left=2cm, right=2cm]{geometry}
\ifpdf
\usepackage[pdftex=true,hyperindex=true,colorlinks=true]{hyperref}
\else
\usepackage[hypertex=true,hyperindex=true,colorlinks=false]{hyperref}
\fi

\newcommand{\cutitle}[4]{
\newpage
\section{#2}
\noindent Nom du projet : #1 \hfill Version #3\\
Spécification du cas d'utilisation : #2 \hfill Date : #4
}
\newcommand{\cutitleshort}[3]{\cutitle{Bureau d'étude 2010/2011}{#1}{#2}{#3}}
\newcommand{\cupresentation}[5]{
\subsection{Présentation}
Acteur initiateur : #1\par
But du cas : #2\par
Préconditions : #3\par
Postconditions : #4\par
Cas d'utilisation inclus : #5
}
\newcommand{\cuscenario}{\subsection{Scénario principal aboutissant au succès}}
\newcommand{\cuscensys}{\hspace{1cm}}
\newcommand{\cualternat}{\subsection{Alternatives}}
\newcommand{\cubesoinsnf}[3]{
\subsection{Besoins non fonctionnels}
A. Fréquence : #1\par
B. Fiabilité : #2\par
C. Performance : #3
}
\newcommand{\cureglesg}{\subsection{Règles de gestion}}
\newcommand{\cutodol}{\subsection{Points ouverts}}

\title{Projet de bureau d'étude : jeu de poker multijoueur\\Cas d'utilisation}
\author{
	Florian \textsc{Canezin},\\
	Matthieu \textsc{Dubet},\\
	Marius \textsc{Moulis},\\
	Régis \textsc{Spadotti},\\
	Guillaume \textsc{Verdier}
}
\date{}

\begin{document}

\maketitle
\clearpage

\tableofcontents

\cutitleshort{CreerPartie}{0.0}{20/10/10}

\cupresentation{Utilisateur}
{ouverture d'une table de jeu selon les paramètres définis (cf règles de gestion).}
{aucune.}
{en attente des connexions d'un nombre défini de joueurs.}
{aucun.}

\cuscenario

1. L'utilisateur demande la création d'une table de jeu

\cuscensys 2. Le système demande les paramètres (cf règles de gestion)

3. L'utilisateur fournit les paramètres

\cuscensys 4. Le système vérifie la validité de ces données et crée la partie

\cualternat

Échec vérification : à l'étape 4, si la validation d'un des paramètres échoue

\cuscensys A.1 GOTO étape 3

\cubesoinsnf{}{}{}

\cureglesg

Les paramètres doivent être valides  :
\begin{itemize}
	\item la cave initiales, à partir de 10\$, sans limite
	\item temps de décision entre 1 et 15 secondes
	\item nombre de joueurs requis pour commencer à jouer, entre 2 et 8
	\item nombre de joueurs maximum à la table, entre 2 à 8
	\item SB et BB de départ
	\item temps entre les blind-up (au moins une minute, sans limite)
	\item schéma d'augmentation des SB, BB et Ante
\end{itemize}

\cutodol

\cutitleshort{RejoindrePartie}{0.0}{20/10/10}

\cupresentation{Utilisateur}
{accèder à une table de jeu de manière à pouvoir jouer, mais sans interrompre une main en cours, donc en devenant spectateur.}
{aucune.}
{\begin{itemize}
	\item spectateur
	\item en attente de fin de main pour pouvoir postuler pour devenir joueur si une place est libre
\end{itemize}}
{aucun.}

\cuscenario

1. L'utilisateur demande l'accès à une table de jeu

\cuscensys 2. Le système vérifie l'existence de la table de jeu et rejoint cette table

\cualternat

Erreur table : à l'étape 2, si la table n'existe pas

\cuscensys A.1 L'utilisateur est averti de la non existence de la table et GOTO 1

Erreur banni : à l'étape 2, si le serveur refuse la connexion car le joueur a été banni

\cuscensys B.1 L'utilisateur est averti et GOTO 1

\cubesoinsnf{}{}{}

\cureglesg

\cutodol

\cutitleshort{DemanderDevenirJoueur}{0.0}{20/10/10}

\cupresentation{Spectateur}
{passer du statut de spectateur à celui de joueur au sein de la table de jeu actuelle, et pouvoir participer à la partie en cours.}
{\begin{itemize}
	\item être spectateur d'une table de jeu ouverte
	\item il n'y a pas de main en cours
\end{itemize}}
{joueur.}
{aucun.}

\cuscenario

1. L'utilisateur demande à devenir joueur

\cuscensys 2. Le système demande au serveur l'autorisation et l'utilisateur devient joueur

\cualternat

Impossible Joueur : à l'etape 2, si le serveur refuse que le spectateur devienne joueur

\cuscensys A.1 Le système prévient l'acteur et celui-ci reste spectateur

\cubesoinsnf{}{}{}

\cureglesg

Le spectateur peut ne pas pouvoir devenir joueur pour plusieurs raisons :
\begin{itemize}
	\item il n'y a pas de place
	\item d'autres spectateurs ont demandé à devenir joueurs avant lui
\end{itemize}

\cutodol

\cutitleshort{QuitterPartie}{0.0}{20/10/10}

\cupresentation{Spectateur ou Joueur}
{se déconnecter volontairement d'une table de jeu.}
{\begin{itemize}
	\item avoir rejoint cette table
	\item si l'acteur initiateur est un joueur, qu'il n'y ait pas de main en cours
\end{itemize}}
{état initial du systeme.}
{aucun.}

\cuscenario

1. L'acteur demande à être déconnecté de la table

\cuscensys 2. Le système revient à son etat initial

\cualternat

Erreur Main : à l'étape 2, si une main est en cours et que l'acteur initiateur y joue

\cuscensys A.1 Le système indique au joueur qu'il lui faut attendre la fin de la main

\cubesoinsnf{}{}{}

\cureglesg

\cutodol

\cutitleshort{EnvoyerMessage}{0.0}{20/10/10}

\cupresentation{Spectateur ou Joueur}
{communiquer avec d'autres joueurs/spectateurs de la table par des messages textuels.}
{avoir rejoint une table.}
{aucune.}
{aucun.}

\cuscenario

1. L'utilisateur demande le post d'un message dans le chat

\cuscensys 2. Le système envoie le message au serveur et l'affiche dans le chat

\cualternat

À l'étape 2, si l'utilisateur n'a pas le droit de poster le message

\cuscensys A.1 Le système l'en informe et n'envoie pas le message

\cubesoinsnf{}{}{}

\cureglesg

L'utilisateur peut ne pas avoir le droit de poster de message si :
\begin{itemize}
	\item il a déjà posté 3 messages d'affilée
	\item il a déjà posté 3 messages en moins d'une minute
	\item le message contient certains mots interdits (injures, par exemple)
\end{itemize}

\cutodol

\cutitleshort{DevenirSpectateur}{0.0}{20/10/10}
\cupresentation{Joueur}
{arrêter de jouer à la table mais observer le déroulement de la partie en cours.}
{\begin{itemize}
	\item avoir rejoint une table
	\item être joueur à cette table
	\item qu'il n'y a pas une main en cours
	\item qu'il y ait au moins 2 autres joueurs à la table
\end{itemize}}
{spectateur.}
{}

\cuscenario

1. L'utilisateur demande à être spectateur

\cuscensys 2. Le système annonce devenir spectateur et le devient

\cualternat

\cubesoinsnf{}{}{}

\cureglesg

\cutodol

\cutitleshort{SeCoucher}{0.0}{20/10/10}

\cupresentation{Joueur}
{arrêter de jouer pour la main en cours.}
{être en train de jouer une main.}
{en attente de la fin de la main courante.}
{}

\cuscenario

1. L'utilisateur demande à se coucher

\cuscensys 2. Le système annonce se coucher et attend la fin de la main en cours

\cualternat

\cubesoinsnf{}{}{}

\cureglesg

\cutodol

\cutitleshort{Suivre}{0.0}{20/10/10}

\cupresentation{Joueur}
{égaliser la mise maximale actuelle au cours d'une main.}
{être en train de jouer une main.}
{en attente du nouveau tour de mise (suivant ou courant à égaliser).}
{}

\cuscenario

1. L'utilisateur annonce qu'il suit la mise actuelle

\cuscensys 2. Le système vérifie que la somme est valide, fait état de l'action et attend le prochaine tour de parole

\cualternat

ERREUR mise : à l'étape 2, si la somme n'est pas valide

\cuscensys A.1 Le système annonce qu'il est tapis.

\cubesoinsnf{}{}{}

\cureglesg 

Le joueur possède l'argent qu'il mise.

\cutodol

\cutitleshort{Relancer}{0.0}{20/10/10}

\cupresentation{Joueur}
{augmenter la mise maximale actuelle au cours d'une main.}
{être en train de jouer une main.}
{en attente du nouveau tour de mise (suivant ou courant à égaliser).}
{}

\cuscenario

1. L'utilisateur demande à relancer d'une certaine somme

\cuscensys 2. Le système vérifie que la somme est valide, fait état de la relance et attend le prochain tour de parole

\cualternat

ERREUR mise : à l'étape 2, si la somme n'est pas valide

\cuscensys A.1 Le système l'annonce au joueur et GOTO 1.

\cubesoinsnf{}{}{}

\cureglesg 

Validation de la mise :
\begin{itemize}
	\item le joueur possède l'argent qu'il mise
	\item la somme est un multiple de la blind
	\item la somme est un multiple de la précédente mise
\end{itemize}

\cutodol

\cutitleshort{FaireTapis}{0.0}{20/10/10}

\cupresentation{Joueur}
{égaliser ou augmenter la mise maximale actuelle en jouant tout son argent.}
{être en train de jouer une main.}
{suivant le résultat de la main, devenir spectateur ou rester joueur.}
{}

\cuscenario

1. L'utilisateur demande à faire tapis 

\cuscensys 2. Le système fait état du tapis et attend le prochain tour de parole

\cualternat

ERREUR mise : à l'étape 2, si la somme n'est pas valide

\cuscensys A.1 Le système l'annonce au joueur et GOTO 1.

\cubesoinsnf{}{}{}

\cureglesg 

Séparation des pots

\cutodol

\cutitleshort{ConsulterStatistiques}{0.0}{20/10/10}

\cupresentation{Joueur}
{obtenir ses propres statistiques ainsi que celles des joueurs jouant à la table.}
{aucune.}
{aucune.}
{aucun.}

\cuscenario

1. L'utilisateur demande l'obtention des statistiques d'un joueur (peut-être lui-même)

\cuscensys 2. Le système lui fournit ces informations

\cualternat

\cubesoinsnf{}{}{}

\cutodol

\end{document}
