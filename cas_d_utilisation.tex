	\documentclass[a4paper]{article}

\usepackage[francais]{babel}
\usepackage[T1]{fontenc}
\usepackage[utf8]{inputenc}
\usepackage[top=2cm, bottom=2cm, left=2cm, right=2cm]{geometry}
\ifpdf
\usepackage[pdftex=true,hyperindex=true,colorlinks=true]{hyperref}
\else
\usepackage[hypertex=true,hyperindex=true,colorlinks=false]{hyperref}
\fi

\newcommand{\cutitle}[4]{
\newpage
\section{#2}
\noindent Nom du projet : #1 \hfill Version #3\\
Spécification du cas d'utilisation : #2 \hfill Date : #4
}
\newcommand{\cutitleshort}[3]{\cutitle{Bureau d'étude 2010/2011}{#1}{#2}{#3}}
\newcommand{\cupresentation}[5]{
\subsection{Présentation}
Acteur initiateur : #1\par
But du cas : #2\par
Préconditions : #3\par
Postconditions : #4\par
Cas d'utilisation inclus : #5
}
\newcommand{\cuscenario}{\subsection{Scénario principal aboutissant au succès}}
\newcommand{\cuscensys}{\hspace{1cm}}
\newcommand{\cualternat}{\subsection{Alternatives}}
\newcommand{\cubesoinsnf}[3]{
\subsection{Besoins non fonctionnels}
A. Fréquence : #1\par
B. Fiabilité : #2\par
C. Performance : #3
}
\newcommand{\cureglesg}{\subsection{Règles de gestion}}
\newcommand{\cutodol}{\subsection{Points ouverts}}

\title{Projet de bureau d'étude : jeu de poker multijoueur\\Cas d'utilisation}
\author{
	Florian \textsc{Canezin},\\
	Matthieu \textsc{Dubet},\\
	Marius \textsc{Moulis},\\
	Régis \textsc{Spadotti},\\
	Guillaume \textsc{Verdier}
}
\date{}

\begin{document}

\maketitle
\clearpage

\tableofcontents

\cutitleshort{CreerPartie}{0.0}{20/10/10}

\cupresentation{Utilisateur}
{ouverture d'une table de jeu selon les paramètres définis (cf règles de gestion).}
{aucune.}
{en attente des connexions d'un nombre défini de joueurs.}
{aucun.}

\cuscenario

1. L'utilisateur demande la création d'une table de jeu

\cuscensys 2. Le système demande le nombre de joueurs pour commencer la partie

3. L'utilisateur donne le nombre de joueurs

\cuscensys 4. Le système vérifie la validité de la valeur saisie et demande le temps d'action ainsi que le temps d'action supplémentaire

5. L'utilisateur fournit ces deux données

\cuscensys 6. Le système vérifie la validité des valeurs saisies et demande la durée entre les augmentations de SB et BB ainsi que le montant de celles-ci

7. L'utilisateur fournit ces deux données

\cuscensys 8. Le système vérifie la validité ces données et crée la partie

\cualternat

Échec <paramètre i> : à l'étape <2i+2>, si la validation du <paramètre i> échoue

\cuscensys A.1 GOTO étape <2i>

\cubesoinsnf{}{}{}

\cureglesg

Les paramètres doivent être valides  :
	* temps de decision entre 0 et 15 secondes.
	* nombre de joueurs maximum à la table compris entre 2 à 8
	* SB,BB de départ
	FIXME(* modalités de changement de SB/BB : par défaut , double à chaque tour de table.?)

\cutodol

\cutitleshort{RejoindrePartie}{0.0}{20/10/10}

\cupresentation{Utilisateur}
{accèder à une table de jeu de manière à pouvoir jouer, mais sans interrompre une main en cours, donc en devenant spectateur.}
{aucune.}
{\begin{itemize}
	\item spectateur
	\item en attente de fin de main pour pouvoir postuler pour devenir joueur si une place est libre
\end{itemize}}
{aucun.}

\cuscenario

1. L'utilisateur demande l'accès à une table de jeu

\cuscensys 2. Le système vérifie l'existence de la table de jeu et rejoins cette table

\cualternat

Erreur table : à l'étape 2, si la vérification de la table échoue

\cuscensys A.1 l'utilisateur est averti de la non existence de la table et GOTO 1

\cubesoinsnf{}{}{}

\cureglesg

FIXME (çà me semble plutot correspondre à creerPartie)
Validité de la table (cf la spécification).)

\cutodol

\cutitleshort{DemanderDevenirJoueur}{0.0}{20/10/10}

\cupresentation{Spectateur}
{passer du statut de spectateur à celui de joueur au sein de la table de jeu actuelle, et pouvoir participer à la partie en cours.}
{\begin{itemize}
	\item être spectateur d'une table de jeu ouverte
	\item il n'y a pas de main en cours
\end{itemize}}
{joueur.}
{aucun.}

\cuscenario

1. L'utilisateur demande à devenir joueur

\cuscensys 2. Le système vérifie le droit de jeu et l'utilisateur devient joueur

\cualternat

Impossible Joueur : à l'etape 2, si la vérification du droit de jeu échoue

\cuscensys A.1 Le système prévient l'acteur et celui-ci reste spectateur

\cubesoinsnf{}{}{}

\cureglesg

Vérification droit de jeu 
FIXME Verifier que le joueur n'a pas été précédemment banni

\cutodol

\cutitleshort{QuitterPartie}{0.0}{20/10/10}

\cupresentation{Spectateur ou Joueur}
{se déconnecter volontairement d'une table de jeu.}
{avoir rejoint cette table.}
{état initial du systeme.}
{aucun.}

\cuscenario

1. L'acteur demande à être déconnecté de la table

\cuscensys 2. Le système revient à son etat initial

\cualternat

\cubesoinsnf{}{}{}

\cureglesg

\cutodol

\cutitleshort{EnvoyerMessage}{0.0}{20/10/10}

\cupresentation{Spectateur ou Joueur}
{communiquer avec d'autres joueurs/spectateurs de la table par des messages textuels.}
{avoir rejoint une table.}
{aucune.}
{aucun.}

\cuscenario

1. L'utilisateur demande le post d'un message dans le chat

\cuscensys 2. Le système envoie le message aux autres joueurs/spectateurs et l'affiche dans le chat

\cualternat

\cubesoinsnf{}{}{}

\cureglesg

FIXME : Verifier que l'utilisateur a le droit d'envoyer ce message selon certains règles:
	* pas plus de 3 messages d'affilés ET en moins de 1 minutes

\cutodol

\cutitleshort{DevenirSpectateur}{0.0}{20/10/10}
\cupresentation{Joueur}
{arrêter de jouer à la table mais observer le déroulement de la partie en cours.}
{avoir rejoint la table. etre joueur de cette table}
{spectateur.}
{}

\cuscenario

1. L'utilisateur demande à être spectateur

\cuscensys 2. Le système annonce devenir Spectateur et le devient

\cualternat

\cubesoinsnf{}{}{}

\cureglesg

FIXME : Si la demande a lieu au cours d'une main, demander une confirmation à l'utilisateur
(pour eviter de se coucher involontairement)
\cutodol

\cutitleshort{SeCoucher}{0.0}{20/10/10}

\cupresentation{Joueur}
{arrêter de jouer pour la main en cours.}
{être en train de jouer une main.}
{en attente de la fin de la main courante.}
{}

\cuscenario

1. L'utilisateur demande à se coucher

\cuscensys 2. Le système annonce se coucher et attend la fin de la main en cours

\cualternat

\cubesoinsnf{}{}{}

\cureglesg

\cutodol

\cutitleshort{Suivre}{0.0}{20/10/10}

\cupresentation{Joueur}
{égaliser la mise maximale actuelle au cours d'une main.}
{être en train de jouer une main.}
{en attente du nouveau tour de mise (suivant ou courant à égaliser).}
{}

\cuscenario

1. L'utilisateur annonce qu'il suit la mise actuelle

\cuscensys 2. Le système vérifie que la somme est valide, fait état de l'action et attend le prochaine tour de parole

\cualternat

ERREUR mise : à l'étape 2, si la somme n'est pas valide

\cuscensys A.1 Le système annonce qu'il est tapis.

\cubesoinsnf{}{}{}

\cureglesg 

Le joueur possede l'argent qu'il mise.

\cutodol

\cutitleshort{Relancer}{0.0}{20/10/10}

\cupresentation{Joueur}
{augmenter la mise maximale actuelle au cours d'une main.}
{être en train de jouer une main.}
{en attente du nouveau tour de mise (suivant ou courant à égaliser).}
{}

\cuscenario

1. L'utilisateur demande à relancer d'une certaine somme

\cuscensys 2. Le système vérifie que la somme est valide, fait état de la relance et attend le prochain tour de parole

\cualternat

ERREUR mise : à l'étape 2, si la somme n'est pas valide

\cuscensys A.1 Le système l'annonce au joueur et GOTO 1.

\cubesoinsnf{}{}{}

\cureglesg 

Validation de la mise :
* le joueur possede l'argent qu'il mise
* la somme est un multiple de la blind
* la somme est un multiple de la précédente mise

\cutodol

\cutitleshort{FaireTapis}{0.0}{20/10/10}

\cupresentation{Joueur}
{egaliser ou augmenter la mise maximale actuelle en jouant tout son argent}
{être en train de jouer une main.}
{suivant le resultat de la main, devenir spectateur ou rester joueur.}
{}

\cuscenario

1. L'utilisateur demande à faire tapis 

\cuscensys 2. Le système fait état du tapis et attend le prochain tour de parole

\cualternat

ERREUR mise : à l'étape 2, si la somme n'est pas valide

\cuscensys A.1 Le système l'annonce au joueur et GOTO 1.

\cubesoinsnf{}{}{}

\cureglesg 

Separation des pots

\cutodol

\cutitleshort{ConsulterStatistiques}{0.0}{20/10/10}

\cupresentation{Joueur}
{obtenir ses propres statistiques ainsi que celles des joueurs jouant à la table.}
{aucune.}
{aucune.}
{aucun.}

\cuscenario

1. L'utilisateur demande l'obtention des statistiques d'un joueur (peut-être lui-même)

\cuscensys 2. Le système lui fournit ces informations

\cualternat

\cubesoinsnf{}{}{}

\cutodol

\end{document}
