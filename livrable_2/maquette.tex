\chapter{Maquette}

\section{Introduction}

Afin de simplifier la présentation de notre maquette de l'interface graphique 
de notre logiciel de poker en P2P, nous allons la diviser selon le modèle MVC 
(Model - View - Controller).\\
Nous aurons 6 fenêtres pour réunir les fonctionnalités de notre application :

\begin{enumerate}
	\item \textbf{Démarrage :} Cette fenêtre sera la fenêtre de démarrage de 
	       notre application et le point central de toutes ses fonctionnalités.
	        Elle devra permettre à l'utilisateur de \textit{créer une partie}, 
	        \textit{rejoindre une partie}, \textit{configurer} le programme 
	        ainsi que de le \textit{quitter}. Elle devra aussi afficher 
	        discrètement des informations élémentaires du logiciel comme par 
	        exemple le langage de développement, les outils utilisées, 
	        le numéro et date de version, les développeurs, 
	        les licences éventuelles... 
	
	\item \textbf{Création de partie :} Cette fenêtre devra permettre à 
	      l'utilisateur de créer une partie facilement. Elle lui proposera de :

	\begin{itemize}
		\item Choisir un nom pour la nouvelle partie.
		\item Un nombre minimum de joueur à attendre pour que la partie commence.
		\item Le nombre de joueur maximum de la table.
		\item Le stack initial des joueurs.
		\item L'intervalle de temps avant l'augmentation des blinds.
		\item Le temps de parole.
		\item Le temps accordé supplémentaire pour une demande de temps.
		\item Le mode d'augmentation des blinds (grand, constant, nul, léger).
		\item Une pré-visualisation du schéma d'augmentation des blinds.
		\item Quel port utiliser (avec une proposition de port par défaut).
		\item Un rappel de son adresse IP.
	\end{itemize}

	\item \textbf{Rejoindre une partie :} Cette fenêtre proposera à l'utilisateur
	      d'entrer une adresse IP d'un serveur de partie ainsi qu'un numéro de 
	      port. Un port par défaut sera proposé à l'utilisateur. Lorsque 
	      l'utilisateur aura cliqué sur `Rejoindre', une 
	      \textit{prévisualisation} de la partie demandée (si elle existe) sera 
	      proposée.
	
	\item \textbf{Prévisualisation :} Cette fenêtre permettra à l'utilisateur, 
	       après que ce dernier est choisi de rejoindre une partie, d'obtenir 
	       une prévisualisation de l'état de la partie qu'il a choisi et ainsi 
	       lui permettre de confirmer ou d'infirmer son choix.\\
	Cette prévisualisation explicitera :

	\begin{itemize}
		\item Le nom de la table.
		\item Le nombre de joueur actuel à la table.
		\item Le pot moyen.
		\item Le nombre moyen de joueur par main.
		\item L'état des blinds.
		\item Un classement des joueurs à la table.
		\item La liste d'attente (des spectateurs).
		\item Le nombre de joueur minimum pour la table.
		\item Le nombre de joueur maximum pour la table.
		\item Le stack initial.
		\item Le blind up.
		\item Le temps de parole.
		\item Le temps accordé supplémentaire pour une demande de temps.
 		\end{itemize} 

		\item \textbf{Configuration :} Cette fenêtre permettra de configurer 
		      l'application. Les préférences de l'utilisateur seront ensuite 
		      enregistrées dans un fichier externe. Avec cette interface, 
		      l'utilisateur pourra configurer deux types de paramètres :

		\begin{itemize}
			\item Les paramètres généraux de l'interface qui lui permettrons 
			      de choisir :

			\begin{itemize}
				\item La langue du logiciel.
				\item La devise utilisée par l'application.
				\item Le thème de l'interface graphique.
				\item Son nom dans le jeu ainsi que son avatar.
			\end{itemize}

			\item Ses préférences, dans lesquelles il décidera :
			
			\begin{itemize}
				\item S'il veut automatiquement cacher sa main lorsqu'il a perdu
				       ou non.
				\item S'il veut automatiquement cacher sa main en cas de victoire
				      par couchage ou non.
				\item S'il veut que le logiciel place automatiquement ses mises 
				      obligatoires pour lui ou non.
				\item S'il veut que la demande de temps soit effectuée à chaque 
				      fois que son temps de décision s'est écoulé ou non.
				\item S'il veut activer l'analyseur (tracker).
			\end{itemize}
			
		\end{itemize}
		
		\item \textbf{Table :} Cette fenêtre sera la fenêtre principale de 
		      l'application et contiendra toutes les fonctionnalités liées au 
		      jeu. Elle affichera : 
		
		\begin{itemize}
			\item La table de jeu, c'est-à-dire : 
			
			\begin{itemize}
				\item Les joueurs (nom, avatar, valeur de son stack) à la table.
				\item Les pots (principal et secondaires) sous forme de jetons. 
				      La valeur de chaque pot pourra être visualisée par 
				      l'utilisateur.
				\item Le tableau.
			\end{itemize}
			
			\item Le classement des 3 premiers joueurs de la table en fonction 
			      de leurs jetons.
			\item Les cartes privatives du \og joueur-utilisateur \fg. 
			\item Les résultats de l'analyseur (tracker) des chances que le 
			      `joueur-utilisateur' à de gagner la main (si ce dernier a 
			       activé la fonctionnalité évidement).
			\item La valeur du stack du \og joueur-utilisateur \fg.
			\item Les action possibles du \og joueur-utilisateur \fg, à savoir : 
			      Se Coucher, Relancer, Miser, Suivre, Passer et cela sous 
			      forme de boutons.
			\item La dernière notification du croupier.
			\item Les fonctionnalités annexes de l'application 
			      (sous forme de slider pour plus d'ergonomie) : 
			
			\begin{itemize}
				\item \textbf{Logs :} Qui affiche l'ensemble des notifications 
				      du croupier.
				\item \textbf{Chat :} Qui permettra à l'utilisateur de chatter 
				      avec les autres joueurs de la table.
				\item \textbf{Statistiques :} Qui affichera les statistiques de 
				      la table et des joueurs.
				\item \textbf{Configuration :} Qui permettra à l'utilisateur de 
				      configurer l'application.
				\end{itemize} 
				
			\end{itemize}
			
		\end{enumerate}

\section{Vue}

\subsection{Les différentes fenêtres de l'application}

L'application sera constituée de 6 fenêtres principales :

\begin{itemize}
	\item Démarrage
	\item Créer partie
	\item Rejoindre partie
	\item Prévisualisation
	\item Table
	\item Table statistiques
	\item Configuration\\
\end{itemize}

L'aperçu graphique de l'application accompagné de chacune des fenêtres est 
disponible, en annexe.

\subsection{Diagramme d'interdépendances entre les différentes fenêtres}
Le diagramme présente la relation de navigation entre les différentes fenêtres 
de l'application. Cette relation est modélisée dans le diagramme par des flèches.
\begin{figure}[ht]
	\centering \includegraphics[width=.75\linewidth]{figures/dependance.pdf}
	\caption{Relation de navigation entre les différentes fenêtres}
\end{figure}

\section{Contrôleur}

\subsection{Introduction}
Nous allons mettre en évidence dans cette partie les différentes contraintes 
que nous imposerons aux composants de notre interface afin d'assurer l'intégrité 
des échanges entre la machine et l'utilisateur ainsi que les actions associées 
aux activations des composants. 

\subsection{Démarrage}

\subsubsection{Contraintes composants}

Il n'y a aucune contrainte sur les composants de cette fenêtre.

\subsubsection{Actions composants}

\begin{itemize}
	\item Le bouton \textit{Créer partie} déclenchera la création d'une nouvelle
	      fenêtre de \textbf{Création de partie}.
	\item Le bouton \textit{Rejoindre partie} déclenchera la création d'une 
	      nouvelle fenêtre de \textbf{Rejoindre partie}.
	\item Le bouton \textit{Configuration} déclenchera la création d'une 
	      nouvelle fenêtre de \textbf{Configuration}.
	\item Le bouton \textit{Quitter} déclenchera la fermeture du programme, 
	      ce qui implique la fermeture des éventuelles connexions réseaux, 
	      la destruction des objets graphiques et l'arrêt du programme.
\end{itemize}

\subsection{Création de partie}

\subsubsection{Contraintes composants}

\begin{itemize}
	\item Le nom de la partie ne doit pas dépasser 64 caractères alphanumériques.
	\item $NombreJoueurMin \in [2 ; 10]$. Le nombre de joueurs minimum aura une 
	      valeur par défaut de 2.
	\item $NombreJoueurMax \in [2 ; 10] \wedge NombreJoueurMin \leq 
	      NombreJoueurMax$. Le nombre de joueur maximum aura une valeur 
	      par défaut de 9.
	\item $StackInitial \in [1 000 ; 100 000] \wedge StackInitial \pmod{1000}
	       = 0 $. Le stack aura une valeur par défaut de 1000.
	\item $BlindUp \in [0 ; 60]$. Le Blind-Up doit être exprimé en minute et 
	      doit être une valeur entière. Le Blind-Up aura une valeur par défaut 
	      de 5. 
	\item $TempsDeParole \in [5; 60]$. Le Temps de parole doit être exprimé en 
	      seconde et doit être une valeur entière. Le Temps de parole aura une 
	      valeur par défaut de 15.
	\item $DemandeDeTemps \in [0; 60]$. La demande de temps doit être exprimée 
	      en seconde et doit être une valeur entière. La DemandeDeTemps aura une
	      valeur par défaut de 45.
	\item Le mode d'augmentation des blinds doit être soit \textit{grand} soit 
	      \textit{léger}, soit \textit{linéaire}, soit \textit{nul}. La valeur 
	      par défaut sera linéaire.
	\item La prévisualisation du schéma d'augmentation des blinds devra être un 
	      tableau de 30 lignes avec 4 colonnes (niveau, SB, BB, ANTE). 
	      Les champs pourront être éditables et seront initialisés en fonction 
	      d'un schéma interne et du stack initial fourni.
	\item $Port \in [1024 ; 65545]$. Le port devra avoir une valeur par défaut de 2000.
\end{itemize}

\subsubsection{Actions composants}

\begin{itemize}
	\item Un bouton \textit{Annuler} détruira la fenêtre courante.
	\item Un bouton \textit{Défaut} permettra de remettre les valeurs 
          par défauts de chaque composant.
	\item Un bouton \textit{Créer Partie} permettra de récolter les valeurs des 
	      composants, regarder s'ils vérifient les contraintes imposées plus haut
	      et ensuite appellera la méthode \texttt{Creer\_Partie()} avec les 
	      paramètres donnés. A la fin, il entrainera la destruction de la fenêtre
	      courante si la création de la partie réussi.
\end{itemize}

\subsection{Rejoindre partie}

\subsubsection{Contraintes composants}

\begin{itemize}
	\item Le champs d'adresse IP devra accepter 4 nombres entiers compris entre 
	      0 et 255. Ce champs aura pour valeur par défaut l'adresse IP du client.
	\item Le numéro de port devra être un nombre entier compris entre 1024 et 65545.
	      Il sera initialisé à 2000.
\end{itemize}

\subsubsection{Actions composants}

\begin{itemize}
	\item Un bouton \textit{Annuler} détruira la fenêtre courante.
	\item Un bouton \textit{Rejoindre} permettra de récolter les valeurs des 
	      composants, regarder s'ils vérifient les contraintes imposées plus haut
	      et ensuite appellera la méthode \texttt{Previsualisation\_Partie()} avec
	      les paramètres donnés. A la fin, il entrainera la destruction de la 
	      fenêtre courante si la prévisualisation réussie.
\end{itemize}

\subsection{Prévisualisation}

\subsubsection{Contraintes composants}
Néant.

\subsubsection{Actions composants}

\begin{itemize}
	\item Un bouton \textit{Retour} détruira la fenêtre courante.
	\item Un bouton \textit{Rejoindre} lancera un appel à la méthode 
	      \texttt{Rejoindre\_Partie()}. Si l'appel réussi, on détruira la fenêtre courante.
\end{itemize}

\subsection{Configuration}

\subsubsection{Contraintes composants}

\begin{itemize}
	\item Le choix de la langue ne prendra que des valeurs finies (Anglais et 
	Français pour l'instant).
	\item Le choix de la devise ne prendra que des valeurs finies (Dollar, Euro
	      et Livre pour l'instant).
	\item Le choix du thème de prendra que des valeurs finies (Thème 1, 2, 3 ou 
	      4 pour l'instant).
	\item Le nom du joueur devra contenir au maximum 64 caractères 
	      alphanumériques. La valeur par défaut sera Joueur.
	\item L'avatar du joueur devra être une image redimensionnée en 100x100 pixels.
	      Les formats possibles seront (GIF, PNG, JPG, BMP et JPEG)
\end{itemize}

\subsubsection{Actions composants}

\begin{itemize}
	\item Un bouton \textit{Annuler} détruira la fenêtre courante.
	\item Un bouton \textit{Rétablir} permettra de remettre les valeurs par défaut.
	\item Un bouton \textit{Sauvegarder} permettra de récolter les valeurs des 
	      composants, de regarder s'ils respectent les contraintes imposées plus
	      haut et ensuite de sauvegarder l'ensemble des préférences dans un fichier
	      de configuration. Si l'enregistrement aboutit, la fenêtre courante est détruite.
\end{itemize}

\subsection{Table}

\subsubsection{Contraintes composants}

\begin{itemize}
	\item Les composants \textit{Configuration}, \textit{Logs}, \textit{Chat} 
	     \& \textit{Statistiques} devront être parfaitement intégrés à 
	     l'interface et de manière ergonomique. Nous envisageons donc 
	     des \og sliders \fg pour chacun de ces trois composants.
	\item La partie \textit{Configuration} reprendra les mêmes composants que 
	      la fenêtre \textit{configuration des préférences du joueur}. 
	\item Les boutons \textit{Miser}, \textit{Relancer} \& \textit{Suivre} 
	      ne pourront pas permettre au joueur de miser plus que son tapis.
	\item Les boutons \textit{Suivre} \& \textit{Miser} s'adapterons en fonction
	      de la situation et pourront devenir respectivement des boutons de 
	      \textit{Passer} et \textit{Relancer}.  
	\item Les boutons \textit{Se Coucher}, \textit{Suivre}, \textit{Passer}, 
	      \textit{Miser} \& \textit{Relancer} ne seront actifs que si c'est au 
	      joueur en question de parler.
\end{itemize}

\subsubsection{Actions composants}

\begin{itemize}
	\item Les boutons \textit{Se Coucher}, \textit{Suivre}, \textit{Passer}, 
	      \textit{Miser} \& \textit{Relancer} provoquerons l'appel des méthodes 
	      qui leur sont associées.
	\item Le composant \textit{Chat} aura un bouton \textit{Envoyer} qui 
	      provoquera l'envoi d'un message aux autres joueurs, un bouton 
	      \textit{Effacer} qui effacera tout le contenu du chat. Un champ de 
	      texte en lecture seule permettra au joueur de lire et de recevoir le 
	      contenu du chat alors qu'un autre permettra de saisir les messages à 
	      envoyer.
	\item Le composant \textit{Statistiques} aura un bouton \textit{Effacer} qui
	      effacera l'ensemble des statistiques enregistrés jusqu'à lors. Les 
	      statistiques seront alors réinitialisées à zéro.
	\item Le composant \textit{Logs} qui affiche les notifications du Croupier 
	      aura lui aussi un bouton \textit{Effacer} qui permettra au joueur 
	      d'effacer l'ensemble des notifications du Croupier.
	\item Tous les pots afficheront leur valeur numérique lorsque le client 
	      passera sa souris dessus.
\end{itemize}

\section{Statistiques}

\subsection{Introduction}

Nous avons décidé de pousser un peu plus loin l'analyse statistique demandée 
dans le sujet de BE et de transformer cette analyse en un module complet gérant 
un maximum d'informations publiques fournises par les joueurs de la table. 
Ces informations seront traitées ensuite par le module et renvoyée au client de 
notre application afin de l'aider à prendre les décisions de son jeux. \\ 
Comme cela représente un assez gros traitement et une grande quantité 
d'informations, nous avons décidé de séparer cette partie du code afin de la 
rendre plus facile à améliorer et à maintenir.\\
Il est à noter que ces statistiques seront totalement locales et éphémères. 
On notera aussi que ces statistiques ont été séparées en deux sous classes : 

\begin{itemize}
	\item Les statistiques générales de la table.
	\item Les statistiques relatives à un joueur.
\end{itemize} 

\subsection{Pour le code}

Un nouveau package JAVA est à prévoir.
Il faudra que chaque événement de la table (Mise d'un joueur, Fin de partie, 
Blind-Up...) provoque l'appel d'une méthode spécifique à l'événement qui s'est 
produit.
Ensuite, cette méthode devra lancer un appel à une méthode 
\texttt{Mise\_A\_Jour()}  qui mettra à jour les statistiques.
Les méthodes \texttt{Initialiser\_Statistiques()} et 
\texttt{Effacer\_Statistiques()} auront respectivement pour mission d'Initialiser
les statistiques à 0 au début de la partie et d'effacer l'ensemble des statistiques
 enregistrés jusque là.

\subsection{Statistiques générales}

Les statistiques générales font références aux facteurs généraux de la table,
 à savoir : 

\begin{itemize}
	\item Le pot moyen de la table.
	\item Nombre moyen de joueurs par main.
	\item L'état des blinds (niveau, SB, BB, ANTE).
	\item Le classement des joueurs à  la table en fonction de leur tapis. 
	     Ce classement mettra en évidence le cheap looser\footnote{Le cheap looser
	    est le joueur qui a le tapis le plus faible de la table.}, le cheap leader
	    \footnote{Le cheap leader est le joueur qui possede le tapis le plus fort 
	    de la table.} ainsi que les killers\footnote{Lorsqu'un joueur élimine un
	     autre joueur de la table, on dit qu'il l'a tué ou \og killé \fg et est 
	    ainsi qualifié de \og killer \fg.} et leur nombre de kill.
	\item Le nombre total de parties distribuées.
	\item Le nombre de Flop, Turn \& River distribués depuis le début de la partie
	      et leur pourcentage par rapport au nombre total de partie distribuées.
	\item Le nombre de pots gagnés sans abattage et son pourcentage par rapport
	      au nombre total de partie distribuées.
	\item Le nombre de Tapis. 
	\item La relance moyenne.
	\item La mise moyenne.
	\item Nombre de tour complet de table.
\end{itemize}

\subsection{Statistiques Joueur}

Les statistiques du joueur sont les statistiques qui lui sont propres, à savoir :

\begin{itemize}
	\item Nombre parties totales jouées.
	\item Nombre de parties engagées (+ pourcentage par rapport au nombre total
	     de parties jouées).
	\item Nombre de parties gagnées (+ pourcentage par rapport au nombre total
	     de parties engagées).
	\item Nombre de parties gagnées sans abattages (+ pourcentage par rapport au
	     nombre total de parties gagnées).
	\item Nombre de Flop, Turn \& River reçu (+ pourcentage par rapport au 
	      nombre de parties engagées).  
	\item La mise moyenne.
	\item La relance moyenne.
	\item Le plus haut pot.
	\item Le plus bas pot.
	\item Nombre de tapis engagés.
	\item Le total des gains.
	\item Le total des pertes.
	\item Pourcentage du nombre total de gains par rapport au total de pertes.
	\item Le statut du joueur : cheap leader, cheap looser, killer ou aucun.
	\item Le nombre de kill.
\end{itemize}

\subsection{Analyseur de main}

En plus des statistiques cités plus haut, nous intégrerons un analyseur de main 
qui permettra au client de savoir en tant réel ses chances de victoires en fonction
 de ses cartes, du nombre de joueurs à la tables et des cartes du tableau. 
